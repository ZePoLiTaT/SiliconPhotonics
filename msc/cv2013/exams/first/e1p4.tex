\section*{Problema 4}
\setcounter{equation}{0}
Demuestre que el conjunto de transformaciones homogéneas de $3 \times 3$ en 2D, tal que
la última fila está dada por $[0,0,1]$ forma un grupo. Este grupo se denomina el grupo afín AL(3).

\subsection*{Solución}
Sean $H$, $G$ e $I$ elementos que pertenecen al grupo afín.
Un grupo se define como un conjunto y un operador binario ($\bullet$) 
sobre los elementos dl conjunto tal que se cumplen las siguientes condiciones:

\begin{enumerate}
\item Conjunto cerrado con respecto al operador :

\begin{equation*}
a,b \in S \leftrightarrow a \bullet b \in S
\end{equation*} 

Al aplicar el operador matricial sobre $H$ y $G$ se obtiene el elemento $J$: 

\begin{align}
\begin{pmatrix}
h_{11} & h_{12} & h_{13} \\
h_{21} & h_{22} & h_{23} \\
0 & 0 & 1
\end{pmatrix}
\times
\begin{pmatrix}
g_{11} & g_{12} & g_{13} \\
g_{21} & g_{22} & g_{23} \\
0 & 0 & 1
\end{pmatrix}
=
\begin{pmatrix}
j_{11} & j_{12} & j_{13} \\
j_{21} & j_{22} & j_{23} \\
0 & 0 & 1
\end{pmatrix}
\end{align} 

Como se puede ver, este nuevo elemento también hace parte del grupo afín.

\item Operador asociativo :
\begin{equation*}
a \bullet (b \bullet c) = (a \bullet b) \bullet c
\end{equation*} 

El operador matricial es asociativo, es decir:

\begin{equation*}
H \times (G \times I) = (H \times G) \times I
\end{equation*} 

y por lo tanto esta propiedad queda demostrada inmediatamente.

\item El elemento identidad pertenece a S $(i \in S)$ :
\begin{equation*}
a \bullet i = i \bullet a = a \; , \forall a \in S
\end{equation*} 

La matriz identidad $I$ es:
\begin{align}
\begin{pmatrix}
1 & 0 & 0 \\
0 & 1 & 0 \\
0 & 0 & 1
\end{pmatrix}
\end{align} 

Claramente se observa que la última fila de esta matriz es $[0,0,1]$ y por lo tanto
pertenece al grupo afín. Adicionalmente, del álgebra lineal se conoce que:

\begin{equation*}
H \times I = I \times H = H
\end{equation*} 

\item La inversa pertence al conjunto:

\begin{equation*}
a \bullet a^{-1} = i \; \forall a \in S
\end{equation*} 

Sea H:
\begin{equation*}
\begin{pmatrix}
h_{11} & h_{12} & h_{13} \\
h_{21} & h_{22} & h_{23} \\
0 & 0 & 1
\end{pmatrix}
\end{equation*} 

\begin{equation}
H^{-1}=\frac{1}{det H} adj H
\label{eq:p40}
\end{equation} 

El determinante de H está dado por:

\begin{align}
det H = |H| &= 0 \times H_{31} + 0 \times H_{32} + 1 \times (h_{11} h_{22} - h_{12} h_{21}) \notag\\
            &= h_{11} h_{22} - h_{12} h_{21}
\label{eq:p41}
\end{align}

La última fila de la matriz adjunta de $H$ está dada por $[H_{13} H_{23} H_{33}]$, donde:

\begin{equation}
\begin{aligned}
H_{13} &= (-1)^{1+3}
\begin{vmatrix}
h_{21} & h_{22} \\
0 & 0
\end{vmatrix}
= 0 \\
H_{23} &= (-1)^{2+3}
\begin{vmatrix}
h_{11} & h_{12} \\
0 & 0
\end{vmatrix}
= 0 \\
H_{33} &= (-1)^{3+3}
\begin{vmatrix}
h_{11} & h_{12} \\
h_{21} & h_{22}
\end{vmatrix}
= h_{11} h_{22} - h_{12} h_{21}
\end{aligned} 
\label{eq:p42}
\end{equation}

Reemplazando \ref{eq:p41} y \ref{eq:p42} en \ref{eq:p40} se tiene que:

\begin{equation}
H^{-1} = 
\begin{pmatrix}
h^{-1}_{11} & h^{-1}_{12} & h^{-1}_{13} \\
h^{-1}_{21} & h^{-1}_{22} & h^{-1}_{23} \\
0 & 0 & 1
\end{pmatrix}
\label{eq:p43}
\end{equation} 

La cual hace parte del grupo afín.

\end{enumerate} 

