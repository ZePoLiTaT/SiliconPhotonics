\section*{Problema 6}
\setcounter{equation}{0}
Demuestre que una transformación afín puede transformar un círculo en una elipse,
pero no puede transformar una elipse en una parábola.

\subsection*{Solución}

Partiendo de los siguientes hechos:
\begin{itemize}
\item La expresión para aplicar una transformación homogénea es:
\begin{equation}
C' = H^{-T} C H^{-1}
\label{eq:p61}
\end{equation} 

\item Como se vio en el punto anterior, la submatriz A de una transformación afín
puede ser expresada como una rotación pura acompañada de un escalamiento anisotrópico.
\begin{equation}
\begin{aligned}
A &= R_\theta D R_{-\theta} R_\phi \\
A^{-1} &= R_{-\phi} R_{\theta} D^{-1} R_{-\theta} \\
A^{-T} &= R_\theta D^-{T} R_{-\theta} R_\phi 
\label{eq:p82}
\end{aligned}
\end{equation} 

Donde $D$ es la matriz de escalamiento anisotrópico y $D^{-1}$ su inversa:

\begin{equation*}
\begin{aligned}
D
\begin{pmatrix}
\lambda_1 & 0 \\ 
0 & \lambda_2
\end{pmatrix} 
\qquad
D^{-1}=D^{-T} = 
\begin{pmatrix}
\frac{1}{\lambda_1} & 0 \\ 
0 & \frac{1}{\lambda_2}
\end{pmatrix} 
\end{aligned}
\end{equation*} 

\item Las cónicas que representan círculos, no se ven afectadas por las rotaciones.

\item Se demostrará el caso no trasladado ya que la traslación no afecta el resultado obtenido.

\item La ecuación general de un círculo en el origen está dada por
\begin{equation*}
x^2+y^2=r^2
\end{equation*}

cuya representación en coordenadas homogéneas es:

\begin{equation*}
C_c=
\begin{pmatrix}
1 & 0 & 0 \\ 
0 & 1 & 0 \\
0 & 0 & -r^2
\end{pmatrix} 
\end{equation*} 

\item La ecuación general de una elipse en el origen y no rotada es:

\begin{equation*}
\frac{x^2}{a^2} + \frac{y^2}{b^2} = 1
\end{equation*} 

Su representación en coordenadas
homogéneas es:

\begin{equation*}
C_{eh}=
\begin{pmatrix}
\frac{1}{a}^2 & 0 & 0 \\ 
0 & \frac{1}{b}^2 & 0 \\
0 & 0 & 1
\end{pmatrix} 
\end{equation*} 

\item La representación homogénea de la elipse se puede ver como la expresada en el punto anterior
aplicando una rotación, así:

\begin{equation}
C_e = R_\theta C_{eh} R_{-\theta}
\label{eq:p83}
\end{equation} 

\end{itemize} 

\subsubsection*{Parte 1}Una transformación afín puede transformar un círculo en una elipse.

El enunciado equivale a demostrar que:
\begin{equation*}
\begin{aligned}
H_A^{-T} C_c H_A{-1} \stackrel{?}{=} C_e
\end{aligned}
\end{equation*} 

Reemplazando \ref{eq:p82} en \ref{eq:p83} tenemos:

\begin{equation*}
\begin{aligned}
R_\theta D^{-T} R_{-\theta} R_\phi C_c R_{-\phi} R_{\theta} D^{-1} R_{-\theta} &\stackrel{?}{=} C_e \\
R_\theta \underbrace{D^{-T} C_c D^{-1}} R_{-\theta} &\stackrel{?}{=} C_e \\
\end{aligned}
\end{equation*} 

\begin{equation*}
\begin{aligned}
D^{-T} C_c D^{-1} &=
\begin{pmatrix}
\frac{1}{\lambda_1} & 0 & 0\\ 
0 & \frac{1}{\lambda_2} & 0\\
0 & 0 & 1
\end{pmatrix} 
\begin{pmatrix}
1 & 0 & 0 \\ 
0 & 1 & 0 \\
0 & 0 & -r^2
\end{pmatrix} 
\begin{pmatrix}
\frac{1}{\lambda_1} & 0 & 0\\ 
0 & \frac{1}{\lambda_2} & 0 \\
0 & 0 & 1
\end{pmatrix} \\
&= 
\begin{pmatrix}
\frac{1}{\lambda_1}^2 & 0 & 0\\ 
0 & \frac{1}{\lambda_2}^2 & 0\\
0 & 0 & -r^2
\end{pmatrix} \\
&= 
\begin{pmatrix}
\frac{1}{a}^2 & 0 & 0\\ 
0 & \frac{1}{b}^2 & 0\\
0 & 0 & 1
\end{pmatrix} = C_{eh}
\end{aligned}
\end{equation*} 

Por lo tanto,

\begin{equation*}
\begin{aligned}
R_\theta
C_{eh}
R_{-\theta}
=
C_e
\end{aligned}
\end{equation*} 

\subsubsection*{Parte 2:} Una transformación afín no puede transformar una elipse en una parábola.

\begin{itemize}
\item La ecuación general de una parábola en el origen y no rotada es:

\begin{equation*}
\begin{aligned}
y^2 &= a x  \\
x^2 &= b y
\end{aligned}
\end{equation*} 

Su representación en coordenadas
homogéneas es:

\begin{equation*}
C_{ph}=
\begin{pmatrix}
1 & 0 & 0 \\ 
0 & a & 0 \\
0 & 0 & 0
\end{pmatrix} 
\end{equation*} 

\item La representación homogénea de la parábola se puede ver como la 
expresada en el punto anterior aplicando una rotación, así:

\begin{equation}
C_p= R_\theta C_{ph} R_{-\theta}
\label{eq:p83}
\end{equation} 

\end{itemize} 

\begin{equation*}
\begin{aligned}
R_\theta D^{-T} R_{-\theta} R_\phi 
C_e 
R_{-\phi} R_{\theta} D^{-1} R_{-\theta} &\stackrel{?}{=} C_p \\
R_\theta D^{-T} 
\underbrace{R_{-\theta} R_\phi R_{\theta'}
C_{eh} 
R_{-\theta'} R_{-\phi} R_{\theta} }
D^{-1} R_{-\theta} &\stackrel{?}{=} C_p \\
R_\theta 
\underbrace{D^{-T}  C_e' D^{-1} }_\text{$C_{ph}'$}
R_{-\theta} &\stackrel{?}{=} C_p \\
C_{ph}' \neq C_{ph}
\end{aligned}
\end{equation*} 

La forma resultante no es como la de una parábola $C_{ph}$.
Adicionalmente, la diagonal resultante de esta operación, se nota que los elementos
quedan multiplicados cada uno por la inversa de los valores propios $\frac{1}{\lambda_1}^2$ y 
$\frac{1}{\lambda_1}^2$ respectivamente. Es decir, el escalamiento anisotrópico afecta ambos
elementos de forma cuadrática y no sólo uno. 
