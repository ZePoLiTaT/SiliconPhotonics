\section*{Problema 3}
\setcounter{equation}{0}

Demuestre que ante una transformación de punto $x' = H x$, una cónica $C$ transforma según:

\begin{equation*}
C'= H^{-T} C H^{-1}
\label{eq:eqname}
\end{equation*} 

\subsection*{Solución}

La expresión de un punto $x$ que pertenece a la cónica $C$ está dada por:
\begin{equation}
x^T C x = 0
\label{eq:p31}
\end{equation} 

y la transformación de un punto dada una homografía $H$ es

\begin{align}
x'&= H x \notag \\
x &= H^{-1} x' \label{eq:p32}
\end{align} 

Reemplazando \ref{eq:p32} en \ref{eq:p31} se tiene:

\begin{align}
(H^{-1} x')^T C  (H^{-1} x') &= 0 \notag \\
x^{'T} \underbrace{H^{-T} C H^{-1}}_\text{$C'$} x' &= 0 \label{eq:p33} 
\end{align} 

Por comparación de \ref{eq:p33} con \ref{eq:p31} se obtiene:

\begin{equation}
C'= H^{-T} C H^{-1}
\label{eq:eqname}
\end{equation} 
