\section*{Problema 8}
\setcounter{equation}{0}
Derive una expresión para la cónica que resulta de intersectar una esfera con un plano 
$\pi_\infty$. Esta cónica se conoce como la cónica absoluta $\Omega_\infty$. 
Describa una aplicación práctica de esta entidad geométrica.

\subsection*{Solución}
La ecuación de una esfera en coordenadas homogéneas es:
\begin{equation}
x_1^2 + x_2^2 + x_3^2 + d x_1 x_4 + e x_2 x_4 + f x_3 x_4 + g x_4^2 = 0
\label{eq:p81}
\end{equation} 

Todo punto $x \in \pi_\infty$ tiene 0 en el último elemento $(x_4=0)$, por lo tanto
la ecuación \ref{eq:p81} queda reducida a:

\begin{align*}
x_1^2 + x_2^2 + x_3^2 = 0 \\
\left(
\begin{array}{c c c}
  x_1 & x_2 & x_3
\end{array}
\right) 
I_{3,3} 
\left(
\begin{array}{c}
    x_1 \\ 
    x_2 \\
    x_3
\end{array}
\right) 
&= 0 
\end{align*} 

Se define entonces la cónica absoluta como todos los puntos imaginarios que
intersectan a cualquier esfera:
\begin{equation*}
\Omega_\infty = I_{3,3}
\end{equation*} 

Una vez identificada $\Omega_\infty$ en una proyectividad, se pueden medir ángulos y
longitudes relativas sobre ésta. Adicionalmente, al ser invariante ante transformaciones
similares, esta cónica se emplea para remover la afinidad.
