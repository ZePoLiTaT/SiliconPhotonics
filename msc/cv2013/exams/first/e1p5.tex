\section*{Problema 5}
\setcounter{equation}{0}
Demuestre que la submatriz superior izquierda de $2 \times 2$ en una transformación afín de 
$3 \times 3$ puede ser descrita como la composición de una rotación pura acompañada por un 
escalamiento anisotrópico.

\subsection*{Solución}

Una transformación afín tiene la siguiente representación:

\begin{equation*}
\begin{pmatrix}
a_{11} & a_{12} & a_{13} \\ 
a_{21} & a_{22} & a_{23} \\ 
0 & 0 & 1
\end{pmatrix} 
= 
\begin{pmatrix}
A & \vec{t} \\ 
\vec{0}^T & 1
\end{pmatrix} 
\end{equation*} 

Donde $A$ es una matriz de $2 \times 2$ no singular. Por lo tanto puede ser descompuesta
mediante SVD de la siguiente forma:

\begin{align*}
A &= U D V^T \\
A &= (U V^T) V D V^T
\end{align*} 

Sea
\begin{align*}
V^T &\triangleq R(\phi) \\
U V^T &\triangleq R(\theta)
\end{align*} 

Donde $R(\phi)$ y $R(\theta)$ son rotaciones en los ángulos $(\phi)$ y $(\theta)$ respectivamente.
La matriz $D$ es una matriz diagonal que contiene los valores propios 
$\lambda_1$ y $\lambda_2$, así:

\begin{equation*}
D = \begin{pmatrix}
\lambda_1 & 0 \\ 
0 &\lambda_2
\end{pmatrix} 
\end{equation*} 

Por lo tanto, la matriz $A$ puede verse como una rotación por un ángulo $\phi$, luego un 
escalamiento anisotrópico dado por los valores propios, una anti-rotación nuevamente
por $\phi$ y finalmente otra rotación por un ángulo $\theta$.


\begin{equation}
A = R(\theta) R(-\phi) D R(\phi)
\label{eq:}
\end{equation} 
