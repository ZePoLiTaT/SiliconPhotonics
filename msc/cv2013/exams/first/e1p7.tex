\section*{Problema 7}
\setcounter{equation}{0}
Derive una expresión para la cónica $C$ que resulta de intersectar una cuádrica
$Q$ con un plano $\pi$. ¿Cómo se relacionan las coordenadas de la cónica $C$ en 2D con
respecto al sistema de coordenadas 3D que comparten $Q$ y $\pi$?.

\subsection*{Solución}
La intersección de una cuádrica $Q$ y un plano $\pi$ es una cónica. Dados 3 puntos
no colineales $A,B,C \in \pi$, todo punto $x$ de $\pi$ está dado por la combinación lineal de
éstos 3:

\begin{align*}
x &= \alpha A + \beta B + \gamma C \\
  &= 
\left(
\begin{array}{c c c}
  A & B & C
\end{array}
\right) 
\left(
\begin{array}{c}
    \alpha \\ 
    \beta \\ 
    \gamma
\end{array}
\right) 
\end{align*} 

Sea $M_{4,3}$ una matriz de transformación que relaciona las coordenadas de un 
punto $p$ en $P2$ a un punto $x$ en $P3$.
\begin{equation*}
x_{4,1} = M_{4,3} p{3,1}
\end{equation*} 

En la intersección, los puntos $x$ también pertenecen a la cónica, es decir:

\begin{align*}
x^T Q x &= 0 \\
(p M)^T Q (p M) &= 0 \\
p^T \underbrace{M^T Q M}_\text{De la forma $x^T C x$} p &= 0\\
\end{align*} 

\begin{equation*}
C = M^T Q M
\end{equation*}

 La relación de la cónica con respecto a la cuádrica se da a través de la matriz $M$.
