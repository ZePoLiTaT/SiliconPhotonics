\section*{Problema 10}
\setcounter{equation}{0}
Demuestre que en 3D dos planos paralelos se intersectan en una línea que está en $\pi_\infty$.

\subsection*{Solución}
Dos planos $\pi_1$ y $\pi_2$ son paralelos si tienen el mismo vector normal $\vec{n_{3,1}}$, 
es decir, pueden ser expresados como:

\begin{equation}
\begin{aligned}
\pi_1 &= \left(
\begin{array}{c}
  \vec{n_{3,1}} \\ 
  d_1
\end{array}
\right) 
\qquad
\pi_2 &= \left(
\begin{array}{c}
  \vec{n_{3,1}} \\ 
  d_2
\end{array}
\right) 
\end{aligned} 
\end{equation}

Donde $d_1 \neq d_2$ son las distancias del origen a los planos 1 y 2 respectivamente.


Sea $X$ un punto que pertenece a la intersección de ambos planos:
\begin{equation*}
X = 
\left(
\begin{array}{c}
  \vec{x_{3,1}} \\ 
  x_4
\end{array}
\right) 
\end{equation*} 

\begin{equation*}
\begin{array}{c c}
\pi_1 X = 0
&
\pi_2 X = 0 \\
\vec{n_{3,1}} x_{3,1} + d_1 x_4 = 0 
&
\vec{n_{3,1}} x_{3,1} + d_2 x_4 = 0\\ 
\end{array} 
\end{equation*} 

Igualando ambas ecuaciones se tiene:
\begin{equation*}
d_1 x_4 = d_2 x_4
\end{equation*} 

La condición $d_1 \neq d_2$ debe cumplirse porque de lo contrario se estaría hablando 
del mismo plano. Por lo tanto, única solución a esta ecuación se da
cuando el componente $x_4$ es 0. Es decir, los puntos de la línea 
formada por la intersección de ambos planos son de la forma:

\begin{equation*}
X = 
\left(
\begin{array}{c}
  \vec{x_{3,1}} \\ 
  0
\end{array}
\right) 
\end{equation*} 

Que pertenecen al plano al infinito.

\begin{equation*}
\pi_\infty X = 
\left(
\begin{array}{c c c c}
  0 & 0 & 0 & 1
\end{array}
\right) 
\left(
\begin{array}{c}
  x_1 \\ 
  x_2 \\ 
  x_3 \\
  0
\end{array}
\right) 
=0
\end{equation*} 
