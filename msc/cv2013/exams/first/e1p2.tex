\section*{Problema 2}
\setcounter{equation}{0}
Demuestre que la línea que une 2 puntos $x$ y $x'$ en 2D está dada por:

\begin{equation}
l=x \times x'
\label{eq:p21}
\end{equation} 

\subsection*{Solución}

Como los puntos $x$ y $x'$ pertenecen a la línea $l$, se cumple que:

\begin {equation}
\begin{aligned}
x^{T} l &= 0 \\
x^{'T} l &= 0
\label{eq:p22}
\end{aligned} 
\end {equation}

Al reemplazar \ref{eq:p21} en \ref{eq:p22} se debe seguir cumpliendo la igualdad:

\begin{align}
x^{T} (x \times x') &= 0 \label{eq:p23} \\
x^{'T} (x \times x') &= 0 \label{eq:p24}
\end{align} 

Sea $x=(x_a, x_b, 1)$ y $x'=(x_c, x_d,1)$.

\begin{equation}
l = 
\begin{vmatrix}
    i & j & k \\
    x_a & x_b & 1 \\
    x_c & x_d & 1
\end{vmatrix}
=
\begin{pmatrix}
    x_b -x_d \\
    x_c -x_a \\
    x_a x_d - x_b x_c \\
\end{pmatrix}
\label{eq:p25}
\end{equation} 

Evaluando el primer punto (reemplazando \ref{eq:p25} en \ref{eq:p23}) se tiene:
\begin{align*}
(x_a, x_b, 1) 
\begin{pmatrix}
    x_b -x_d \\
    x_c -x_a \\
    x_a x_d - x_b x_c \\
\end{pmatrix}
= 0
\end{align*} 

\begin{align*}
\cancel{x_a x_b} -\cancel{x_a x_d} + \cancel{x_b x_c} - \cancel{x_a x_b} + \cancel{x_a x_d} - \cancel{x_b x_c} &=0\\
0 &= 0
\end{align*} 

Ahora para el segundo punto se tiene:
\begin{align*}
(x_c, x_d, 1) 
\begin{pmatrix}
    x_b -x_d \\
    x_c -x_a \\
    x_a x_d - x_b x_c \\
\end{pmatrix}
= 0
\end{align*} 

\begin{align*}
\cancel{x_b x_c} -\cancel{x_c x_d} + \cancel{x_c x_d} - \cancel{x_a x_d} + \cancel{x_a x_d} - \cancel{x_b x_c} &=0\\
0 &= 0
\end{align*} 


