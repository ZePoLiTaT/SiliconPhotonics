\section*{Problem 1}

Let $T(n)$ be the temperature, in degrees Fahrenheit, of a cup of tea initially at a 
temperature of $180^\circ$F that stays at a constant room temperature of $80^\circ$F.
After one minute, the tea temperature has decreased to $175^\circ$F.

How hot is the tea after 30 minutes?

Hint: Use Newton's law of Cooling.
\begin{equation}
T(n-1) - T(n) = k[T(n) - 80]
\label{eq:newton}
\end{equation} 

\subsection*{Solution}

From (\ref{eq:newton}) we find K:

\begin{equation*}
\begin{aligned}
T(n-1) - T(n) &= k[T(n) - 80]\\
175 - 180 &= k[180 - 80]\\
k&= \frac{-1}{20}
\end{aligned}
\end{equation*} 

Now we solve the difference equation using the Z-Transform:

\begin{equation*}
\begin{aligned}
T(0) &= 180^\circ \\
T(1) &= 175^\circ \\
\end{aligned}
\end{equation*} 

\begin{equation*}
\begin{aligned}
zT(z) - zT(0) - T(z) &= k T(z) - 80\frac{z}{z-1} \\
T(z)[z -1 -k] &= 180z - 80k \frac{z}{z-1}\\
T(z) &= z \frac{[180(z-1) - 80k]}{z-1}\\
\frac{T(z)}{z} &= \frac{180z - 180 - 80k}{(z-1)(z-1-k)}\\
&= \frac{A}{z-1} + \frac{B}{Z-1-k}
\end{aligned}
\end{equation*} 

Solving by partial fractions we find 

\begin{equation*}
\begin{aligned}
A &= \frac{80k}{k} = 90 \\
B &= \frac{180+180k-180-80k}{1+k-1} = 100
\end{aligned}
\end{equation*} 

Therefore:

\begin{equation*}
\begin{aligned}
T(z) &= 80 \frac{z}{z-1} + 100 \frac{z}{z-1-k} \\
T(n) &= 80 u(n) + 100 (1+k)^n \\
&= 80 u(n) + 100 (\frac{19}{20})^n
\end{aligned}
\end{equation*} 

When $n=30$ we have:

\begin{equation*}
\begin{aligned}
T(30) &= 80 + 100(\frac{19}{20})^{30} \\
 &\approx 101.464
\end{aligned}
\end{equation*} 

Code implementation of the system:

\zcodec{../ztransform/main.cpp}{C++ Implementation}


