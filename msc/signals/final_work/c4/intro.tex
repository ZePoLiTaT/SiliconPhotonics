When we move into a discrete time domain, the Fourier Transform can be calculated
using the DTFT or Discrete Time Fourier Transform
which is defined as \cite{kamen2000fundamentals}:

\begin{equation}
X(\omega) = \mathfrak{F}\{ x[n] \} = \displaystyle\sum_{- \infty}^{\infty} x[n] e^{-j \omega n}
\label{eq:c41}
\end{equation} 

As can be seen in (\ref{eq:c41}), the frecuency domain obtained is continuous. 
However, in order to manipulate (save/restore) such information in a computer,
the frecuency should also be discretized.
For time-frecuency discrete domains, the DFT is defined over the
time interval from $n=0$ to $n=N$ as:

\begin{equation}
X_k = \displaystyle\sum_{n=0}^{N-1} x[n] e^{-j \frac{2 \pi k}{N} n}
\label{eq:c42}
\end{equation} 

where the frecuency $\omega$ was redefined as $\omega = \frac{2 \pi k}{N}$.

The following function was implemented to calculate the $k = [0,1,..N]$
values for the Transform. Additionally, an extra parameter
$N$ was added to force the input array to be of a bigger size.
This will be used in some problems to extend the size of a discrete
function $x[n]$ assuming that it fades to 0 for the missing values.

\zcodemat{sources/dft2.m}{Fourier Discrete Transform with 0 padding implementation}

