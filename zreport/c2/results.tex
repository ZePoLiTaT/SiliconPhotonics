\subsection{Resultados}

Para obtener el tamaño de la muestra necesario para una confiabilidad del 
95\% se empleó la siguiente ecuación: 

\begin{equation}
n=\frac{z^2 s^2}{B^2}
\label{eq:muestra}
\end{equation} 

Donde $n$ es el tamaño de la muestra y $s$ la desviación estándar.
El nivel de confianza dado por $z$ se estableció a un valor de 1.96 que corresponde
a un 95\% de confianza. Por último el parámetro $B$ que indica la precisión,
se estableció según la variable a medir. Para el consumo de energía, este valor
fluctúa entre 0.1\% y 0.8\%, mientras que para la latencia se toma en el orden de 
los nanosegundos.

En las tablas de \ref{tb:sn512} a \ref{tb:st65k} se muestran los valores obtenidos en cada simulación
para las variables de consumo de energía y latencia y el valor $n$ obtenido en cada caso:

\begin{table}[H]
\centering
\begin{tabular}{|c|c|c|c|c|}
\hline
 &E&P&E&P\\
\hline
Random-Seed&All Energy $(J)$&All Energy $(J)$&Latency ($\mu$ s)&Latency ($\mu$ s)\\
\hline
6&0.0670822&0.00795576&0.0691798&0.0792249\\
5&0.0671684&0.00795162&0.0683049&0.0779179\\
4&0.0670545&0.00797981&0.0695929&0.0772176\\
2&0.0670536&0.00795009&0.0700826&0.0790122\\
10&0.0670933&0.00796371&0.0681599&0.0778642\\
8&0.0670919&0.00795244&0.0691727&0.0840341\\
\hline
stdev&4.19633E-05&1.13364E-05&0.000739481&0.002480388\\
mean&0.06709065&0.007958905&0.069082133&0.079211817\\
z&1.96&1.96&1.96&1.96\\
p&0.0015&0.0015&0.03&0.03\\
B&0.000100636&1.19384E-05&0.002072464&0.002376355\\
\hline
n&0.667950099&3.463963085&0.48909428&4.185322852\\
\hline
\end{tabular}
\caption{Datos Simulación App: Neighboor, PkgSize: 512B}
\label{tb:sn512}
\end{table}

\begin{table}[H]
\centering
\begin{tabular}{|c|c|c|c|c|}
\hline
 &E&P&E&P\\
\hline
Random-Seed&All Energy $(J)$&All Energy $(J)$&Latency ($\mu$ s)&Latency ($\mu$ s)\\
\hline
6&0.0672651&0.0079509&0.0865564&0.127786\\
5&0.0670529&0.0079602&0.0894102&0.132535\\
4&0.0670557&0.00796118&0.087135&0.132628\\
2&0.0671679&0.00796183&0.0870683&0.132774\\
10&0.0671439&0.00794885&0.0892598&0.133439\\
\hline
stdev&8.81681E-05&6.20164E-06&0.001342664&0.002289528\\
mean&0.0671371&0.007956592&0.08788594&0.1318324\\
z&1.96&1.96&1.96&1.96\\
p&0.0012&0.0012&0.03&0.03\\
B&8.05645E-05&9.54791E-06&0.002636578&0.003954972\\
\hline
n&4.600953054&1.620723506&0.996242993&1.287411017\\
\hline
\end{tabular}
\caption{Datos Simulación App: Random, PkgSize: 512B}
\label{tb:sr512}
\end{table}

\begin{table}[H]
\centering
\begin{tabular}{|c|c|c|c|c|}
\hline
 &E&P&E&P\\
\hline
Random-Seed&All Energy $(J)$&All Energy $(J)$&Latency ($\mu$ s)&Latency ($\mu$ s)\\
\hline
6&0.0671366&0.00795344&0.116493&0.197834\\
5&0.0671364&0.00796231&0.115827&0.20161\\
4&0.0671169&0.00795815&0.115762&0.198954\\
2&0.0670515&0.00795665&0.116925&0.202891\\
10&0.0672859&0.00795283&0.11856&0.205117\\
8&0.0671018&0.0079642&0.11602&0.201612\\
\hline
stdev&7.88881E-05&4.61203E-06&0.001057811&0.002638301\\
mean&0.067138183&0.00795793&0.116597833&0.201336333\\
z&1.96&1.96&1.96&1.96\\
p&0.0012&0.0012&0.03&0.03\\
B&8.05658E-05&9.54952E-06&0.003497935&0.00604009\\
\hline
n&3.68326535&0.896053751&0.351321474&0.732949518\\
\hline
\end{tabular}
\caption{Datos Simulación App: BitReverse, PkgSize: 512B}
\label{tb:sb512}
\end{table}

\begin{table}[H]
\centering
\begin{tabular}{|c|c|c|c|c|}
\hline
 &E&P&E&P\\
\hline
Random-Seed&All Energy $(J)$&All Energy $(J)$&Latency ($\mu$ s)&Latency ($\mu$ s)\\
\hline
6&0.0671167&0.00797317&0.0898645&0.131713\\
5&0.0670928&0.00797042&0.0994328&0.153952\\
4&0.067287&0.00795941&0.0989025&0.15703\\
2&0.0671297&0.00794928&0.0796521&0.111226\\
10&0.0671326&0.00796399&0.097888&0.158137\\
8&0.0670367&0.00795338&0.0888971&0.133294\\
7&0.0673762&0.00798954&0.090947&0.132685\\
\hline
stdev&0.000119549&1.36029E-05&0.007129387&0.017359803\\
mean&0.067167386&0.007965599&0.092226286&0.139719571\\
z&1.96&1.96&1.96&1.96\\
p&0.0015&0.0015&0.095&0.095\\
B&0.000100751&1.19484E-05&0.008761497&0.013273359\\
\hline
n&5.408864344&4.979186199&2.543664501&6.571129653\\
\hline
\end{tabular}
\caption{Datos Simulación App: HotSpot, PkgSize: 512B}
\label{tb:sh512}
\end{table}

\begin{table}[H]
\centering
\begin{tabular}{|c|c|c|c|c|}
\hline
 &E&P&E&P\\
\hline
Random-Seed&All Energy $(J)$&All Energy $(J)$&Latency ($\mu$ s)&Latency ($\mu$ s)\\
\hline
6&0.0671112&0.00794923&0.0736731&0.0837278\\
5&0.0670484&0.00796436&0.0702405&0.0840095\\
4&0.0671063&0.00796425&0.0707252&0.0889056\\
2&0.067085&0.00795086&0.0715833&0.0927564\\
10&0.0670837&0.0079514&0.0709125&0.08952\\
8&0.0670981&0.00794945&0.0704214&0.0836357\\
7&0.0670697&0.00795191&0.0692727&0.0849632\\
\hline
stdev&2.19061E-05&6.77157E-06&0.001382019&0.003603739\\
mean&0.067086057&0.007954494&0.070975529&0.086788314\\
z&1.96&1.96&1.96&1.96\\
p&0.001&0.001&0.035&0.035\\
B&6.70861E-05&7.95449E-06&0.002484144&0.003037591\\
\hline
n&0.409615903&2.783977714&1.189014215&5.40704914\\
\hline
\end{tabular}
\caption{Datos Simulación App: Tornado, PkgSize: 512B}
\label{tb:st512}
\end{table}


\begin{table}[H]
\centering
\begin{tabular}{|c|c|c|c|c|}
\hline
 &E&P&E&P\\
\hline
Random-Seed&All Energy $(J)$&All Energy $(J)$&Latency ($\mu$ s)&Latency ($\mu$ s)\\
\hline
6&0.0673445&0.00799389&1.51514&0.51567\\
5&0.0671345&0.00799492&1.51138&0.528793\\
4&0.0671509&0.00799738&1.50786&0.522043\\
2&0.0672506&0.00800407&1.50719&0.534181\\
\hline
stdev&9.75049E-05&4.57715E-06&0.003659849&0.008048858\\
mean&0.067220125&0.007997565&1.5103925&0.52517175\\
z&1.96&1.96&1.96&1.96\\
p&0.0015&0.0015&0.03&0.03\\
B&0.00010083&1.19963E-05&0.045311775&0.015755153\\
\hline
n&3.59239214&0.559247838&0.025062029&1.002618005\\
\hline
\end{tabular}
\caption{Datos Simulación App: Random, PkgSize: 65536B}
\label{tb:sr65k}
\end{table}


\begin{table}[H]
\centering
\begin{tabular}{|c|c|c|c|c|}
\hline
 &E&P&E&P\\
\hline
Random-Seed&All Energy $(J)$&All Energy $(J)$&Latency ($\mu$ s)&Latency ($\mu$ s)\\
\hline
6&0.0671338&0.00797883&1.39792&0.328061\\
5&0.0672169&0.0080181&1.39465&0.322767\\
4&0.0671059&0.00797706&1.40607&0.312189\\
2&&0.00798905&&0.323677\\
10&&0.00800833&&0.337259\\
8&&0.00798299&&0.311023\\
\hline
stdev&5.77423E-05&1.69321E-05&0.00588121&0.009881777\\
mean&0.0671522&0.007992393&1.399546667&0.322496\\
z&1.96&1.96&1.96&1.96\\
p&0.0017&0.0017&0.03&0.03\\
B&0.000114159&1.35871E-05&0.0419864&0.00967488\\
\hline
n&0.982837116&5.966009944&0.075375164&4.007661738\\
\hline
\end{tabular}
\caption{Datos Simulación App: Neighboor, PkgSize: 65536B}
\label{tb:sn65k}
\end{table}


\begin{table}[H]
\centering
\begin{tabular}{|c|c|c|c|c|}
\hline
 &E&P&E&P\\
\hline
Random-Seed&All Energy $(J)$&All Energy $(J)$&Latency ($\mu$ s)&Latency ($\mu$ s)\\
\hline
6&0.0672625&0.00799928&1.55446&0.611643\\
5&0.0672618&0.00801075&1.55022&0.605117\\
4&0.0672407&0.00799567&1.55219&0.610232\\
2&0.0671883&0.00800271&1.55416&0.615228\\
\hline
stdev&3.48504E-05&6.44184E-06&0.001968627&0.004191135\\
mean&0.067238325&0.008002103&1.5527575&0.610555\\
z&1.96&1.96&1.96&1.96\\
p&0.001&0.001&0.007&0.007\\
B&6.72383E-05&8.0021E-06&0.010869303&0.004273885\\
\hline
n&1.032032731&2.489566182&0.126018875&3.69428032\\
\hline
\end{tabular}
\caption{Datos Simulación App: BitReverse, PkgSize: 65536B}
\label{tb:sb65k}
\end{table}


\begin{table}[H]
\centering
\begin{tabular}{|c|c|c|c|c|}
\hline
 &E&P&E&P\\
\hline
Random-Seed&All Energy $(J)$&All Energy $(J)$&Latency ($\mu$ s)&Latency ($\mu$ s)\\
\hline
6&0.0671994&0.007995&1.56549&0.52534\\
5&0.0671903&0.00800166&1.59177&0.550099\\
4&0.067383&0.00803194&1.59952&0.571698\\
2&0.067201&0.00799349&1.56815&0.502018\\
10&0.0672379&0.00800522&1.59104&0.554626\\
8&0.0671213&0.00800309&1.58954&0.526473\\
7&0.0674615&0.00801929&1.57188&0.525672\\
\hline
stdev&0.000120675&1.38231E-05&0.013577873&0.023394081\\
mean&0.067256343&0.008007099&1.582484286&0.536560857\\
z&1.96&1.96&1.96&1.96\\
p&0.0014&0.0014&0.035&0.035\\
B&9.41589E-05&1.12099E-05&0.05538695&0.01877963\\
\hline
n&6.30995691&5.841412563&0.230866389&5.96142471\\
\hline
\end{tabular}
\caption{Datos Simulación App: HotSpot, PkgSize: 65536B}
\label{tb:sh65k}
\end{table}


\begin{table}[H]
\centering
\begin{tabular}{|c|c|c|c|c|}
\hline
 &E&P&E&P\\
\hline
Random-Seed&All Energy $(J)$&All Energy $(J)$&Latency ($\mu$ s)&Latency ($\mu$ s)\\
\hline
6&0.0671693&0.00798253&1.44903&0.373905\\
5&0.0671009&0.00799408&1.42758&0.404685\\
4&0.0671575&0.00800383&1.44217&0.377759\\
2&0.067091&0.00798413&1.43348&0.384779\\
10&0.0671379&0.00799219&1.43278&0.398339\\
8&0.0671538&0.00798319&1.43929&0.388029\\
7&0.0670885&0.00798815&1.43149&0.390422\\
\hline
stdev&3.41615E-05&7.66216E-06&0.007354642&0.010836262\\
mean&0.067128414&0.007989729&1.436545714&0.388274\\
z&1.96&1.96&1.96&1.96\\
p&0.0008&0.0008&0.025&0.025\\
B&5.37027E-05&6.39178E-06&0.035913643&0.00970685\\
\hline
n&1.554509071&5.520398642&0.161107709&4.787563253\\
\hline
\end{tabular}
\caption{Datos Simulación App: Tornado, PkgSize: 65536B}
\label{tb:st65k}
\end{table}

Una vez ejecutadas las simulaciones, se realizaron pruebas hipótesis de diferencia
de medias de la tabla \ref{tb:mvi} para validar o no la hipótesis de que la 
red de interconexión nanofotónicas híbridas con topología P-Mesh es mejor
que su equivalente electrónica E-Mesh.
Las variables que se utilizaron como indicador 
de un mejor sistema fueron el consumo de energía y la latencia del sistema.


\begin{table}[H]
\centering
\begin{tabular}{|c|c|c|c|}
\hline
Caso & Hipótesis & Estadístico de Prueba & Criterio de Rechazo \\
\hline 
$\sigma_1^2 = \sigma_2^2$ &
\specialcell{
    $H_0:\mu_1 = \mu_2$ \\
    $H_A:\mu_1 > \mu_2$ 
} &
\specialcell{
    $S_p = \sqrt{\frac{(n_1-1)S_1^2 + (n_2-1)S_2^2}{n_1+n_2-2}}$ \\ \\
    $t_0=\frac{\hat{x_1} - \hat{x_2}}{S_p\sqrt{\frac{1}{n_1}+\frac{1}{n_2}}} $ \\ \\
    $v= n_1 + n_2 - 2$
}& 
$t_0 > t_{\alpha,v}$ 
\\ 
\hline
$\sigma_1^2 \not= \sigma_2^2$ &
\specialcell{
    $H_0:\mu_1 = \mu_2$ \\
    $H_A:\mu_1 > \mu_2$ 
} &
\specialcell{
    $t_0=\frac{\hat{x_1} - \hat{x_2}}{S_p\sqrt{\frac{S_1^2}{n_1}+\frac{S_2^2}{n_2}}} $ \\ \\
    $v= \frac{ (\frac{S_1^2}{n_1} -  \frac{S_2^2}{n_2}) }
	     {\frac{(\frac{S_1^2}{n_1})^2}{n_1-1} + \frac{(\frac{S_2^2}{n_2})^2}{n_2-1}} $
}& 
$t_0 > t_{\alpha,v}$ 
\\ 
\hline
\end{tabular}
\caption{Estadísticos para Diferencia de Medias. Fuente: \cite{gutierrez2003analisis} }
\label{tb:mvi}
\end{table} 

Según el $n$ obtenido para cada caso en las tablas \ref{tb:sn512} a \ref{tb:st65k}, 
se procedió a realizar el análisis de las 5 aplicaciones
en 2 subgrupos según el tamaño de los mensajes enviados en cada momento.
Adicionalmente, en cada caso, se hizo un analisis de varianza \cite{trejos2004} 
para saber cual de las 2 fórmulas de \ref{tb:mvi} utilizar.

Como el nivel de confianza realizado para la prueba de hipótesis fue del 0.95,
se tiene un valor de $\alpha$ = 0.05. Por lo tanto sólo las pruebas que presenten un valor
inferior a alfa en el valor P se podrá decir con una confiabilidad del 95\% que la
red nanofotónica híbrida es mejor en alguna de las 2 variables que la electrónica.

La prueba de hipótesis y la alterna para las 2 variables de medición se puede ver en la 
tabla \ref{tb:h0ha}. En ambos casos, se busca rechazar la $H_0$, 
es decir confirmar que la red nanofotónica híbrida
se comporta mejor en términos de consumo de energía o lantencia que la electrónica. 


\begin{table}[H]
\centering
\begin{tabular}{|c|c|c|}
\hline
$H_0$ & $H_a$ & Variable\\ 
\hline
$CE_E = CE_P$ & $CE_E > CE_P$ & Consumo de Energía (CE) \\
\hline
$L_E = L_P$ & $L_E > L_P$ & Latencia (L) \\
\hline
\end{tabular}
\caption{My caption here}
\label{tb:h0ha}
\end{table} 

\subsubsection{Paquetes Pequeños (512B)}

En las tablas \ref{tb:eall512} y \ref{tb:lall512} se sintetizan 
las medias del consumo de energía y de
latencia para los mensajes pequeños.

\begin{table}[H]
\centering
\begin{tabular}{|c|c|c|c|}
\hline
&Electrónica&Híbrida Fotónica\\
\hline
Neighboor&0.06709065&0.007958905\\
Random&0.0671371&0.007956592\\
BitReverse&0.067138183&0.00795793\\
HotSpot&0.067167386&0.007965599\\
Tornado&0.067086057&0.007954494\\
\hline
\end{tabular}
\caption{Datos promedio Consumo de Energía $(J)$, PkgSize: 512B}
\label{tb:eall512}
\end{table}


\begin{table}[H]
\centering
\begin{tabular}{|c|c|c|c|}
\hline
&Electrónica&Híbrida Fotónica\\
\hline
Neighboor&0.069082133&0.079211817\\
Random&0.08788594&0.1318324\\
BitReverse&0.116597833&0.201336333\\
HotSpot&0.092226286&0.139719571\\
Tornado&0.070975529&0.086788314\\
\hline
\end{tabular}
\caption{Datos promedio Latencia $(\mu s)$, PkgSize: 512B}
\label{tb:lall512}
\end{table}

Se puede observar gráficamente que el consumo de energía de la red
nanofotónica híbrida es mucho menor que el de la red netamente electrónica.
Sin embargo, la variable de latencia muestra que para este tipo de mensajes
la red fotónica híbrida se comporta peor que la red electrónica.

\begin{figure}[H]
\caption{Comparación Consumo de Energía, PkgSize: 512B}
\centering
\includegraphics[width=1.0\textwidth,natwidth=483,natheight=256]{figs/E512.png}
\label{fig:e512}
\end{figure} 

\begin{figure}[H]
\caption{Comparación Latencia, PkgSize: 512B}
\centering
\includegraphics[width=1.0\textwidth,natwidth=483,natheight=256]{figs/L512.png}
\label{fig:l512}
\end{figure} 

Los resultados de las pruebas de hipótesis se presentan a continuación:
 
\begin{table}[H]
\centering
\begin{tabular}{|c|c|c|c|}
\hline
&Valor P&Rechaza $H_0$ ?\\
\hline
Neighboor&7.29052E-32&TRUE\\
Random&2.21789E-23&TRUE\\
BitReverse&2.85143E-29&TRUE\\
HotSpot&1.42176E-32&TRUE\\
Tornado&3.3077E-41&TRUE\\
\hline
\end{tabular}
\caption{Resultados Prueba Hipótesis para la variable Consumo de Energía, PkgSize: 512B}
\label{tb:ettest512}
\end{table}

En la tabla \ref{tb:ettest512} se encontró que en todos los casos se presentó una
respuesta positiva. Esto se puede concluir debido a que el valor P para
en todos los resultados fue mayor a 0.05 y por lo tanto se puede concluir que 
la red nanofotónica híbrida sí es mejor en términos de consumo de potencia
que la red electrónica analizada para tamaños de mensajes pequeños.


\begin{table}[H]
\centering
\begin{tabular}{|c|c|c|c|}
\hline
&Valor P&Rechaza $H_0$ ?\\
\hline
Neighboor&0.99999883&FALSE\\
Random&1&FALSE\\
BitReverse&1&FALSE\\
HotSpot&0.999989&FALSE\\
Tornado&0.9999999&FALSE\\
\hline
\end{tabular}
\caption{Resultados Prueba Hipótesis para la variable Latencia, PkgSize: 512B}
\label{tb:lttest512}
\end{table}

En la tabla \ref{tb:lttest512} se puede confirmar lo observado
gráficamente para la variable de 
latencia. En los paquetes de mensajes pequeños,
no hay evidencias suficientes para afirmar que la red fotónica
es mejor que la electrónica. Es más, en todos los casos se comportó peor.

Este resultado se puede entender desde el punto de vista de la red híbrida,
ya que al necesitar un subplano de control electrónico para hacer el \textit{setup} del
camino antes de enviar los datos por el subplano fotónico. Cuando los mensajes son
pequeños, el sobrecosto de realizar este trabajo sobrepasa en gran medida el 
beneficio de tener una red de transmisión fotónica \cite{shacham2008photonic} y por lo tanto no
logra el rendimiento esperado.

\subsubsection{Tamaño de Paquetes Grande (65kB)}

En las tablas \ref{tb:eall65k} y \ref{tb:lall65k} se sintetizan 
las medias del consumo de energía y de
latencia para los mensajes grandes (65kB).

\begin{table}[H]
\centering
\begin{tabular}{|c|c|c|c|}
\hline
&Electrónica&Híbrida Fotónica\\
\hline
Neighboor&0.0671522&0.007992393\\
Random&0.067220125&0.007997565\\
BitReverse&0.067238325&0.008002103\\
HotSpot&0.067256343&0.008007099\\
Tornado&0.067128414&0.007989729\\
\hline
\end{tabular}
\caption{Datos promedio Consumo de Energía $(J)$, PkgSize: 65kB}
\label{tb:eall65k}
\end{table}


\begin{table}[H]
\centering
\begin{tabular}{|c|c|c|c|}
\hline
&Electrónica&Híbrida Fotónica\\
\hline
Neighboor&1.399546667&0.322496\\
Random&1.5103925&0.52517175\\
BitReverse&1.5527575&0.610555\\
HotSpot&1.582484286&0.536560857\\
Tornado&1.436545714&0.388274\\
\hline
\end{tabular}
\caption{Datos promedio Latencia $(\mu s)$, PkgSize: 65kB}
\label{tb:lall65k}
\end{table}

Nuevamente se puede observar gráficamente que el consumo de energía de la red
nanofotónica híbrida es mucho menor que el de la red netamente electrónica.
Sin embargo, en este caso, la variable de latencia SI muestra que 
para este tipo de mensajes la red fotónica híbrida se comporta mucho mejor 
que la red electrónica.

\begin{figure}[H]
\caption{Comparación Consumo de Energía, PkgSize: 65kB}
\centering
\includegraphics[width=1.0\textwidth,natwidth=483,natheight=306]{figs/E65k.png}
\label{fig:e65k}
\end{figure} 

\begin{figure}[H]
\caption{Comparación Latencia, PkgSize: 65kB}
\centering
\includegraphics[width=1.0\textwidth,natwidth=483,natheight=306]{figs/L65k.png}
\label{fig:l65k}
\end{figure} 

\begin{table}[H]
\centering
\begin{tabular}{|c|c|c|c|}
\hline
&Valor P&Rechaza $H_0$ ?\\
\hline
Neighboor&2.427440e-22 TRUE&TRUE\\
Random&1.05729E-17&TRUE\\
BitReverse&2.41872E-20&TRUE\\
HotSpot&1.57746E-32&TRUE\\
Tornado&5.3047E-39&TRUE\\
\hline
\end{tabular}
\caption{Resultados Prueba Hipótesis para la variable Consumo de Energía, PkgSize: 65kB}
\label{tb:ettest65k}
\end{table}

En la tabla \ref{tb:ettest65k} se puede observar en todos los casos una
respuesta positiva debido a que el valor P para
todos los resultados fue mayor a 0.05. Por lo tanto, también se puede concluir que 
la red nanofotónica híbrida sí es mejor en términos de consumo de potencia
que la red electrónica analizada para tamaños de mensajes grandes.

\begin{table}[H]
\centering
\begin{tabular}{|c|c|c|c|}
\hline
&Valor P&Rechaza $H_0$ ?\\
\hline
Neighboor&7.08709E-14&TRUE\\
Random&5.37024E-10&TRUE\\
BitReverse&3.01947E-11&TRUE\\
HotSpot&2.99509E-16&TRUE\\
Tornado&8.35414E-21&TRUE\\
\hline
\end{tabular}
\caption{Resultados Prueba Hipótesis para la variable Latencia, PkgSize: 65kB}
\label{tb:lttest65k}
\end{table}

En la tabla \ref{tb:lttest65k} también se encontró en todos los casos una
respuesta positiva. Es decir que para tamaños de mensajes grandes, el sobrecosto
de la subred de control es insignificante en cuanto a las mejoras obtenidas
en latencia por el subplano fotónico de transmisión de datos.

