\subsection{Conclusiones y Trabajos Futuros}

\begin{itemize}
\item Para tamaños de mensajes grandes, la red de interconexión nanofotónica
híbrida C-Mesh se comporta mejor en términos de consumo de energía y latencia
que la red electrónica C-Mesh.
\item  Para tamaños de mensajes grandes, la red de interconexión nanofotónica
híbrida C-Mesh se comporta mejor únicamente en términos de consumo de energía 
que la red electrónica C-Mesh.
\item  En los paquetes de mensajes pequeños,
no hay evidencias suficientes para afirmar que la red fotónica
es mejor que la electrónica en la variable de latencia debido a que
el sobrecosto de la subred de control electrónica realizar 
sobrepasa en gran medida el beneficio de tener una red de transmisión fotónica 
y por lo tanto no logra el rendimiento esperado.
\item Se plantea como trabajos futuros, evaluar otras topologías en
la red de interconexión, así como ampliar la prueba ejecutando también
el subgrupo de aplicaciones científicas.
\item Debido a los resultados obtenidos para mensajes pequeños en términos de 
latencia, sería ideal probar otros tipos de redes implementadas
diferentes a las híbridas conmutadas, es decir, que no tengan un subplano
de control electrónico. Por ejemplo la TDM arbitrada por longitud de onda
descrito en \cite{hendry2011time} o \cite{vantrease2008corona}.
\item Fortalecer los benchmark con los que se analiza el rendimiento de las redes
electrónicas y nanofotónicas en PhoenixSim, abarcando un espectro más amplio
de aplicaciones que involucren áreas como Machine Learning, Software de Base de Datos,
Computación Gráfica y Juegos.
\end{itemize} 
