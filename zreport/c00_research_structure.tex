\section{Resumen}
Las tendencias actuales de los procesadores muestran que en un periodo corto de tiempo, 
alcanzaremos los cientos de núcleos en un solo chip. 
A medida que la cantidad de estos núcleos aumenta, los requerimientos de ancho de banda 
de las redes de interconexión que permiten la comunicación interna entre estos y 
hacia la memoria, se incrementa. A medida que se incrementa el ancho de banda en 
sistemas de interconexión electrónicos, la latencia y la disipación de poder, 
se ven impactadas considerablemente, por lo que dicha solución se vuelve no viable.

La fusión del campo de la fotónica con la nanotecnología denominado 
nanofotónica, ha permitido el desarrollo de dispositivos
basados en silicio de altas prestaciones y bajo consumo, ya que pueden 
ser producidos usando las técnicas ya existentes de manufacturación de semiconductores, 
y gracias a que el silicio actualmente es utilizado como el componente base 
en la mayoría de circuitos, es posible crear dispositivos híbridos en los cuales 
se integran componentes tanto electrónicos como ópticos en un solo microchip. 
Tomando en cuenta estos últimos avances, las redes de interconexión ópticas 
en un chip o NoCs nano ópticas sobre silicio, ya ha sido conceptualizada, 
permitiendo superar las actuales limitaciones de su equivalente electrónico 
en los chips de multiprocesamiento o CMP.

En este proyecto se pretende explorar y evaluar diferentes diseños de 
arquitecturas que permitan analizar el impacto de las NoCs ópticas en 
las futuras generaciones de CMPs, ya que son de vital importancia para 
lograr el aumento en el rendimiento manteniendo al mismo tiempo la eficiencia 
en elconsumo de potencia.

Al finalizar, se tendrán los análsis comparativos en términos de ancho de banda, 
potencia y latencia de algunas de las redes de interconexión más conocidas 
sobre diferentes conjuntos de pruebas. La aplicación de este conocimiento 
científico orientado al área de arquitectura de computadores y a la nanotecnología, 
permitirá la creación de nuevos diseños que aporten a la solución de los
problemas de rendimiento vs potencia, apuntando a mejorar la competitividad 
nacional mediante la generación de producción intelectual en dicho aspecto.

\section{Planteamiento del Problema}
Las redes de interconexión electrónicas, tienen un impacto directo en 
la limitación de potencia, ancho de banda y latencia de los
chips multiprocesadores (CMPs) actuales. 
Estas limitaciones sumadas a la inhabilidad de escalar eficientemente a cientos de
núcleos, la presenta como una solución poco viable a largo plazo.


\section{Justificación}
Para lograr escalar eficientemente los sistemas de multiprocesamiento, las tecnologías 
actuales deben superar grandes dificultades, tanto en la complejidad adicional 
que representa la programación en paralelo, las limitaciones de ancho de banda y en la
minimización de la cantidad de potencia que es desperdiciada en el sistema de comunicación. 
En esta última en especial, se evidencia el límite al que están llegando las 
tecnologías electrónicas actuales, al consumir casi la mitad \ref{} de la potencia dinámica
de todo el sistema, haciendo de vital importancia la exploración de otras alternativas.

El empleo de la tecnología de punta que usa la nanofotónica sobre silicio, 
se presenta como una solución prometedora a este
problema, ya que no solo permite el flujo de grandes cantidades de 
datos en una misma línea de transmisión, sino que tiene una facilidad de 
integración con la microelectrónica actual manteniendo al mismo tiempo los 
bajos costes de fabricación, gracias a que se pueden reutilizar las técnicas 
de manufactura de semiconductores tradicionales. Es por esto que el tema es foco de
investigación no solo de grandes compañías microelectrónicas (Intel, IBM, HP, etc) 
sino también de la comunidad académica (Columbia, Purdue, MIT entre otras), 
quienes han logrado demostrar experimentalmente las piezas fundamentales que 
permitirían su materialización en un futuro no muy lejano.

Para lograr este objetivo, se debe permitir primero a los arquitectos de computadores, 
a través de la simulación, explorar de forma ágil nuevos diseños de arquitecturas 
de interconexión y evaluar el rendimiento holístico a través de mediciones 
precisas en los aspectos de capacidad de procesamiento, consumo de potencia y escalabilidad.

Por otro lado, y aunque de forma relativamente reciente, también se ha destacado 
la importancia del fortalecimiento y aplicación de este tipo de tecnologías convergentes 
en el contexto nacional. En 2004, la nanotecnología fue reconocida por Colciencias como
una de las 8 áreas estratégicas del conocimiento para impulsar el desarrollo 
competitivo del país. En el documento "Visión Colombia II Centenario: 2019" 
se confirma la continuidad del apoyo institucional y financiero 
a las iniciativas relacionadas con dichas áreas. 
Finalmente, es mencionada también dentro del 
"Plan Nacional de Desarrollo Científico, Tecnológico y de Innovación 2007-2019" 
de Colciencias y el Departamento Nacional de Planeación, 
como un área perteneciente a la "Nueva Ciencia" y "Tercera Revolución Industrial" 
en donde se debe disminuir la brecha tecnológica con respecto a otros países.

\section{Revisión Bibliográfica}
\subsection*{Antecedentes}

\section{Objetivos}
\subsection{Objetivo General}
Simular y evaluar redes de interconexión nanofotónicas híbridas 
sobre silicio para chips multiprocesadores.

\subsection{Objetivos Específicos}
\begin{itemize}
\item Caracterizar los componentes básicos de las redes nanofotónicas sobre silicio.
\item Explorar las topologías de interconexión más usadas en los CMPs electrónicos actuales.
\item Mapear las topologías de interconexión del dominio electrónico al dominio nanofotónico híbrido sobre silicio.
\item Simular cada una de las topologías en los dominios electrónico y nanofotónico híbrido sobre silicio.
\item Evaluar y comparar las redes en términos de ancho de banda, latencia y potencia.
\end{itemize} 

\section{Hipótesis}
¿Son las redes de interconexión nanofotónicas híbridas para las
topologías Mesh y Torus más eficientes en términos de consumo de potencia,
ancho de banda y latencia que sus pares electrónicos?

\section{Metodología}
Este proyecto se plantea como una investigación aplicada donde se busca 
encontrar soluciones del estado del arte en el área de arquitectura de 
computadores para la problemática del consumo de recursos de las redes de 
interconexión electrónicas actuales en los chips multiprocesadores. 

Resulta entonces conveniente la utilización de una metodología estructural, 
dividiendo la investigación en cuatro fases que dan solución a la 
problemática propuesta:

\begin{itemize}
\item Fase 0: Base Teórica
En esta etapa se hace un estudio de los fundamentos teóricos tanto físicos
como matemáticos que rigen el comportamiento de los dispositivos que soportan
los componentes que forman una red de interconexión nanofotónica sobre silicio. 

\item Fase 1: Caracterización de los componentes pasivos y activos de las redes nanofotónicas. 
Es necesario analizar el comportamiento, las técnicas de manufactura y 
los requisitos de funcionamiento de los siguientes componentes nanofotónicos sobre 
silicio: guías de onda recta, guías de onda dobladas, cruces de guía de onda 
y resonadores en anillo, así como las simulaciones corespondientes de cada elemento 
para diferentes longitudes de onda y potencia de entrada.

En esta etapa, se hará uso del software CAD para el diseño y simulación que se ejecutará 
sobre un servidor especializado para este propósito. 

\item Fase 2: Análisis de Topologías de Interconexión
Análisis y la exploración de las topologías de interconexión electrónicas 
actuales que se desean evaluar en el alcance de este proyecto: Mesh y Torus 
en el dominio de los CMPs.
Una vez obtenida la información de las topologías en el dominio electrónico, 
se debe proceder a hacer el análisis del correspondiente mapeo de dichas topologías 
al dominio nanofotónico teniendo en cuenta las restricciones o 
ventajas de los componentes nanofotónicos pasivos y activos estudiados en la etapa
anterior. 

\item Fase 3: Simulación 

\end{itemize} 

