\section{Resumen}
Las tendencias actuales de los procesadores muestran que en un periodo corto de tiempo, 
alcanzaremos los cientos de núcleos en un solo chip. 
A medida que la cantidad de estos núcleos aumenta, los requerimientos de ancho de banda 
de las redes de interconexión que permiten la comunicación interna entre estos y 
hacia la memoria, se incrementa. A medida que se incrementa el ancho de banda en 
sistemas de interconexión electrónicos, la latencia y la disipación de poder, 
se ven impactadas considerablemente, por lo que dicha solución se vuelve no viable.

La fusión del campo de la fotónica con la nanotecnología denominado 
nanofotónica, ha permitido el desarrollo de dispositivos
basados en silicio de altas prestaciones y bajo consumo, ya que pueden 
ser producidos usando las técnicas ya existentes de manufacturación de semiconductores, 
y gracias a que el silicio actualmente es utilizado como el componente base 
en la mayoría de circuitos, es posible crear dispositivos híbridos en los cuales 
se integran componentes tanto electrónicos como ópticos en un solo microchip. 
Tomando en cuenta estos últimos avances, las redes de interconexión ópticas 
en un chip o NoCs nano ópticas sobre silicio, ya ha sido conceptualizada, 
permitiendo superar las actuales limitaciones de su equivalente electrónico 
en los chips de multiprocesamiento o CMP.

En este proyecto se empleó una metodología estructural encaminada a resolver
la siguiente hipótesis: ¿Es la red de interconexión nanofotónica híbrida para la
topologías C-Mesh más eficiente en términos de consumo de potencia
 y latencia que su equivalente electrónico?. 

Para lograr este objetivo se identificaron entonces 2 grandes etapas: 

La primera correspondió
a la visualización del problema desde el punto de vista del comportamiento físico de
los dispositivos que componen la red de interconexión nanofotónica. 

En ésta se simuló
una de las estructuras más importantes en Silicon Photonics: el anillo resonador. Se 
obtuvieron los resultados de las longitudes de onda resonantes y de la separación
entre los modos para las configuraciones de filtro Notch y filtro AddDrop y adicionalmente
se ejecutaron las pruebas sobre 2 software FDTD diferentes.


La segunda tuvo un enfoque con un nivel de abstracción superior, extrayendo
sólo algunos parámetros de la primera etapa como base para un análisis holístico de rendimiento,
latencia y potencia del sistema.

En esta etapa se simularon 2 redes de interconexión con topología C-Mesh para 5 aplicaciones
sintéticas diferentes y con tamaños de mensajes pequeños y grandes. Con los tamaños grandes
se comprobó la hipótesis, es decir, la red nanofotónica híbrida se comportó mejor en
términos de potencia y latencia que la electrónica. Sin embargo, para tamaños de paquetes
pequeños el sobrecosto en el que incurre la red híbrida debido al subplano de control,
hace que el beneficio en consumo de potencia se vea opacado por una latencia mayor
que la electrónica.

La aplicación de este conocimiento 
científico orientado al área de arquitectura de computadores y a la nanotecnología, 
permitirá la creación de nuevos diseños que aporten a la solución de los
problemas de rendimiento vs potencia, apuntando a mejorar la competitividad 
nacional mediante la generación de producción intelectual en dicho aspecto.

%TODO Resumen de metodología 
%TODO Resumen de resultados y conclusiones
\begin{comment}
En este proyecto se evalúa el comportamiento de los elementos básicos que conforman
los dispositivos fotónicos. Esta caracterización a nivel físico, permite tener un
mayor control sobre los parámetros con los que se simula su comportamiento a nivel
de un sistema completo en una capa de abstracción superior.

También se analizan diferentes diseños de 
arquitecturas que permiten analizar el impacto de las NoCs ópticas en 
las futuras generaciones de CMPs. 

Este impacto es de vital importancia para 
lograr el aumento en el rendimiento manteniendo al mismo tiempo la eficiencia 
en el consumo de potencia.

Al finalizar, se tendrán los análsis comparativos en términos de ancho de banda, 
potencia y latencia de algunas de las redes de interconexión más conocidas 
sobre diferentes conjuntos de pruebas. La aplicación de este conocimiento 
científico orientado al área de arquitectura de computadores y a la nanotecnología, 
permitirá la creación de nuevos diseños que aporten a la solución de los
problemas de rendimiento vs potencia, apuntando a mejorar la competitividad 
nacional mediante la generación de producción intelectual en dicho aspecto.
\end{comment}

\section{Planteamiento del Problema}
Las redes de interconexión electrónicas, tienen un impacto directo en 
la limitación de potencia, ancho de banda y latencia de los
chips multiprocesadores (CMPs) actuales. 
Estas limitaciones sumadas a la inhabilidad de escalar eficientemente a cientos de
núcleos, la presenta como una solución poco viable a largo plazo.


\section{Justificación}
Para lograr escalar eficientemente los sistemas de multiprocesamiento, las tecnologías 
actuales deben superar grandes dificultades, tanto en la complejidad adicional 
que representa la programación en paralelo, las limitaciones de ancho de banda y en la
minimización de la cantidad de potencia que es desperdiciada en el sistema de comunicación. 
En esta última en especial, se evidencia el límite al que están llegando las 
tecnologías electrónicas actuales, al consumir casi la mitad \cite{Magen2004} de la 
potencia dinámica de todo el sistema, haciendo de vital importancia la 
exploración de otras alternativas.

El empleo de la tecnología de punta que usa la nanofotónica sobre silicio, 
se presenta como una solución prometedora a este
problema, ya que no solo permite el flujo de grandes cantidades de 
datos en una misma línea de transmisión, sino que tiene una facilidad de 
integración con la microelectrónica actual manteniendo al mismo tiempo los 
bajos costes de fabricación, gracias a que se pueden reutilizar las técnicas 
de manufactura de semiconductores tradicionales. Es por esto que el tema es foco de
investigación no solo de grandes compañías microelectrónicas (Intel, IBM, HP, etc) 
sino también de la comunidad académica (Columbia, Purdue, MIT entre otras), 
quienes han logrado demostrar experimentalmente las piezas fundamentales que 
permitirían su materialización en un futuro no muy lejano.

Debido a que los dispositivos ópticos son fundamentalmente diferentes de 
las tecnologías de interconexión electrónicas convencionales, se requieren nuevas 
metodologías y herramientas que permitan a 
los arquitectos de computadores, evaluar el rendimiento holístico del sistema a 
través de mediciones precisas en los aspectos de capacidad de procesamiento, 
consumo de potencia y escalabididad.
Todo esto, incorporando de forma acertada las diferencias físicas 
en cuanto al comportamiento en la fotónica \cite{Chan2010}.

Por otro lado, y aunque de forma relativamente reciente, también se ha destacado 
la importancia del fortalecimiento y aplicación de este tipo de tecnologías convergentes 
en el contexto nacional. En 2004, la nanotecnología fue reconocida por Colciencias como
una de las 8 áreas estratégicas del conocimiento para impulsar el desarrollo 
competitivo del país. En el documento "Visión Colombia II Centenario: 2019" 
se confirma la continuidad del apoyo institucional y financiero 
a las iniciativas relacionadas con dichas áreas. 
Finalmente, es mencionada también dentro del 
"Plan Nacional de Desarrollo Científico, Tecnológico y de Innovación 2007-2019" 
de Colciencias y el Departamento Nacional de Planeación, 
como un área perteneciente a la "Nueva Ciencia" y "Tercera Revolución Industrial" 
en donde se debe disminuir la brecha tecnológica con respecto a otros países.

\section{Revisión Bibliográfica}
\subsection*{Antecedentes}
La naturaleza creativa inherente al ser humano, le ha impulsado a través de los tiempos
hacia la búsqueda y análisis de información. El desarrollo de las tecnologías orientadas
no solo a la manipulación de dichos datos, sino a la facilidad en el intercambio de los
mismos, se han convertido en parte indispensable de la sociedad actual, la cual exige
cada vez, mayores capacidades y mejores tiempos de respuesta en los mismos.
Para satisfacer estos requisitos, y como consecuencia directa, la industria de los
microprocesadores ha realizado constantes avances desde los años 70, en donde era
más que suficiente una unidad de procesamiento (single-core). A partir de allí, se ha
llegado a los sistemas actuales, en donde varios núcleos de una misma unidad (multi-
core) realizan el trabajo. La tendencia de dicha industria, en un futuro a mediano-largo
plazo, es el incremento de estos núcleos al orden de cientos o miles, en donde su
cantidad tiene una proyección de crecimiento en un factor de 2x cada 18 meses \cite{Magen2004},
dando surgimiento a una nueva ley similar a la ley de Moore para los transistores.

Sin embargo, y para preservar las ventajas esperadas al tener cada vez más núcleos de
procesamiento, se deben no solo mantener los bajos costos de fabricación, sino superar
tres grandes dificultades \cite{Asanovic2007} relacionadas con: el aumento en la 
complejidad por la programación en paralelo, el limitado ancho de banda hacia los dispositivos de
entrada/salida y la disipación de poder de la red de interconexión entre núcleos, siendo
esta la responsable del consumo de alrededor del $50\%$ \cite{Bergman2007} del 
poder total del chip.

Los grandes avances sobre la nueva tecnología de nanofotónica sobre silicio, han
demostrado los componentes necesarios \cite{Chan2011} para la creación de redes que permitan
conectar componentes tanto dentro como fuera del chip, no solo aprovechando las
altas velocidades de transmisión y bajo consumo de potencia, sino también
conservando el uso de las técnicas ya existentes de manufacturación de
semiconductores. Esto facilita la integración con la tecnología actual, manteniendo
viable los costes de fabricación a nivel masivo, comparado con los sistemas ópticos
convencionales.

No obstante, el uso de nanofotónica sobre silicio aplicado a las redes de interconexión
debe seguir nuevas metodologías y herramientas que permitan un diseño preciso y
acertado en un tiempo apropiado \cite{Chan2010b}. Dentro de las herramientas disponibles, se
destacan las siguientes:

El proyecto PhoenixSim \cite{Chan2010} 
(Photonic and Electronic Network Integration and eXecution
Simulator) de la Universidad de Columbia, ha capturado las características físicas los
componentes fotónicos y electrónicos usados en este tipo de redes, sobre un software
de simulación que soporta la interacción de decenas de núcleos. Esto permite no solo la
simulación y comparación de redes de interconexión en ambos dominios, sino también
la creación de sistemas híbridos que aprovechen las ventajas que ofrece cada uno.

Las plataformas de prototipado BEE3 \cite{davis2009bee3}, 
desarrolladas en la Universidad de Stanford,
permiten la completa emulación de CMPs usando las redes de interconexión eléctrica
convencionales. En este sistema, Watts \cite{watts2013} propone la implementación de módulos de
lógica que presenten un comportamiento equivalente a los dispositivos de nanofotónica
sobre silicio, permitiendo así alcanzar tiempos de simulación que permitan una
evaluación dinámica de múltiples arquitecturas NoCs.


\section{Objetivos}
\subsection{Objetivo General}
Simular y evaluar redes de interconexión con topología C-Mesh nanofotónicas híbridas 
sobre silicio para chips multiprocesadores.

\subsection{Objetivos Específicos}
\begin{itemize}
\item Caracterizar los componentes básicos de los dispositivos que conforman la
capa de interconexión en las redes nanofotónicas sobre silicio.
\item Comprender los parámetros que gobiernan, a nivel físico, los dispositivos 
de una red de interconexión nanofotónica para su manipulación conciente en
 un nivel de abstracción superior o de sistema.
\item Explorar las topologías de interconexión más usadas en los CMPs electrónicos actuales.
\item Simular la topologías C-Mesh con aplicaciones sintéticas y
para tamaños pequeños y grandes de mensajes en los dominios electrónico y 
nanofotónico híbrido sobre silicio.
\item Evaluar y comparar las redes en términos de latencia y potencia.
\end{itemize} 

\section{Hipótesis}
¿Es la red de interconexión nanofotónica híbrida para la
topologías C-Mesh más eficiente en términos de consumo de potencia
 y latencia que su equivalente electrónico?

\section{Metodología}
Este proyecto se planteó como una investigación aplicada donde se buscaba 
encontrar soluciones del estado del arte en el área de arquitectura de 
computadores para la problemática del consumo de recursos de las redes de 
interconexión electrónicas actuales en los chips multiprocesadores. 

Resultó entonces conveniente la utilización de una metodología estructural, 
en la cual se identificaron 2 grandes etapas: 

La primera corresponde
a la visualización del problema desde el punto de vista del comportamiento físico de
los dispositivos que componen la red de interconexión nanofotónica. 
La segunda tiene un enfoque con un nivel de abstracción superior, extrayendo
sólo algunos parámetros de la primera etapa como base para un análisis holístico de rendimiento,
latencia y potencia del sistema.

La importancia de realizar ambas etapas radica en el conocimiento adquirido en la primera
 permite una manipulación conciente de los parámetros que son insumo de la segunda. 


\subsection{Etapa 1}
\begin{itemize}
\item Fase 1: Base Teórica
En esta etapa se hizo un estudio de los fundamentos teóricos tanto físicos
como matemáticos que rigen el comportamiento de los dispositivos que soportan
los componentes que forman una red de interconexión nanofotónica sobre silicio. 

\item Fase 2: Caracterización de los componentes pasivos y activos de las redes nanofotónicas. 
Es necesario analizar el funcionamiento de los siguientes componentes 
nanofotónicos sobre silicio: guías de onda recta, guías de onda dobladas, cruces de guía de onda 
y anillos resonadores.

Finalmente, se ejecutarán las simulaciones corespondientes a 
2 de los elementos activos más relevantes que
emplean anillos resonadores: filtro Notch y filtro AddDrop.

En esta etapa, se usarán 2 softwares de simulación que correrán 
sobre dos servidores especializados para este propósito. 
\end{itemize} 

\subsection{Etapa 2} 
\begin{itemize}
\item Fase 3: Análisis de Topologías de Interconexión
Se analizará la topología de interconexión electrónica 
que se desean evaluar en el alcance de este proyecto: Mesh Concentrada o C-Mesh
para 5 aplicaciones
sintéticas diferentes y con tamaños de paquetes grandes y pequeños
.

Una vez obtenida la información de las topologías en el dominio electrónico, 
se procederá a simular estas topologías híbridas y se evaluará su rendimiento
midiendo los resultados en términos de potencia y latencia.

\end{itemize} 

\section{Resultados}

Debido a la estructura metodológica sobre la que se desarrolló esta investigación,
los resultados se presentarán en los 2 capítulos siguientes, correspondientes a 
los resultados obtenidos en cada una de las fases mencionadas.
