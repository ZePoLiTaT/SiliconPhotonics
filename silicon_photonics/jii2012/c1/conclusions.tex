Se pueden observar diferencias en cuanto al valor teórico esperado en las
condiciones de resonancia para todas las simulaciones,
debido a que se asumió que el anillo no presentaba pérdidas $(\alpha = 1)$ en
el cálculo teórico de la transmitancia.

Sin embargo, la que más difiere de todas inclusive en el FSR esperado 
es la simulación mostrada en la figura \ref{fig:meep_res_n}. Esto se puede deber a
que esta simulación se realizó sólamente en 2 dimensiones y por lo tanto
deja de lado la interacción de la onda con el materia de Sílica SiO2.


\subsection{Conclusiones y Trabajos Futuros}
\begin{itemize}
\item Las frecuencias de resonancia del anillo pueden ser manipuladas mediante
modificaciones en el diámetro del anillo, el gap o el índice de refracción. Este
último es de gran importancia por ser la base del funcionamiento algunos dispositivos
moduladores en Silicon Photonics donde se aplica un campo
magnético, generando un efecto plasmónico que altera el índice y por lo tanto
los modos de resonancia.
\item Debido a que se asumió que el anillo no presentaba pérdidas $(\alpha = 1)$ en
el cálculo teórico de la transmitancia tanto del filtro Notch como del filtro
AddDrop, se pueden apreciar diferencias en las condiciones de resonancia
de la teoría con respecto a lo obtenido en la práctica.
\item Las diferencias en el $FSR$ obtenido en las simulaciones realizadas en Meep
con respecto a las realizadas en Lumerical, pueden ser explicadas debido a la 
diferencia en las dimensiones de ambas simulaciones. Mientras que en Lumerical 
se realizó una simulación 3D de ambos filtros, en Meep la mayoría fueron realizadas
en 2D a diferencia del último caso. Este escenario permitió corroborar esta teoría
ya que la simulación Meep 3D del filtro Notch fue muy similar a la simulación del mismo
filtro en la plataforma Lumerical.
\item Como trabajo futuro se plantea realizar nuevas simulaciones para los dos
filtros modificando el diámetro del anillo para visualizar 
la alteración en el FSR y su uso como filtro selectivo o filtro de banda ancha.
\item Se propone también, para un diámetro fijo, realizar simulaciones alterando 
el índice de refracción del anillo para simular el efecto plasmónico que 
sucede dentro del modulador y que hace que se desplacen las frecuencias de resonancia
de un estado On a Off.
\end{itemize} 
